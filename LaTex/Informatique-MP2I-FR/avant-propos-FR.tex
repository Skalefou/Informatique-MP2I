\section{Avant-propos}
\label{subsec:avant-propos}


Cette feuille d'exercices a pour objectif de présenter une série d'exercices d'informatique que peut rencontrer un étudiant en CPGE MP2I.
Ces exercices ont la particularité d'inclure des notions mathématiques avancées que peut rencontrer un élève.
J'ai pris la liberté d'inclure d'autres exercices, des énoncés de projet à réaliser, des exercices venant de certains concours pour apporter
une certaine profondeur dans les différents enseignements.
Cette série d'exercices est donc destinée à certains de mes amis pour leur proposer une manière différente et exigeante d'apprendre l'informatique
et les mathématiques. Elle n'est donc pas destinée à des étudiants de prépa MP2I ni à des mathématiciens, et de ce fait, je m'inspirerai énormément
d'exercices de différents professeurs de CPGE que j'ai trouvés sur internet. Au début de chaque exercice, je donnerai la source et les références
utilisées pour la création de chaque exercice de cette feuille. Pour ceux qui demandent un niveau avancé en mathématiques, une partie cours sera proposée
avant le début de chaque énoncé pour une bonne compréhension. Cependant, je n'incorporerai aucun cours relatif à l'informatique, j'inclurai des
références dans la préface pour pouvoir étudier les différents langages et leur paradigme.

La difficulté des exercices sera graduelle. Je commencerai par un échauffement qui simulera l'entraînement que suit un élève entre le lycée et
son premier jour de cours, et je finirai par des exercices de concours. Une correction de chaque exercice sera proposée sur le dépôt GitHub.
Les exercices seront d'abord pour l'échauffement en Python puis ils seront principalement en C, OCaml et SQL.

J'espère que vous apprécierez cette feuille et que tu prendras plaisir à la parcourir autant que j'ai pris plaisir à l'écrire.